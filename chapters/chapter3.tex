\chapter{概述}
在本节中,我们正式定义我们的问题,并阐述这些挑战及相应的解决方案。

\section{A. 问题描述}
令 $M$ 表示一个由 $F$ 个文件组成的恶意软件程序,其中第 $i$ 个文件 ($1 ≤ i ≤ F$) 包含 $G$ 个函数,记作 \{$f_{1}^{i}$, $f_{2}^{i}$, ..., $f_{G}^{i}$\}。对于由语言模型(LLM)应用的给定转换策略 $s$,我们的目标是生成一个恶意软件变种 $M_{s}$,其中第 i 个文件包含使用策略 s 生成的修改后函数 \{$\hat{f_{1}^{i}}$, $\hat{f_{2}^{i}}$, ..., $\hat{f_{j}^{i}}$\},同时保留未修改的函数 \{$f_{j+1}^{i}$, ..., $f_{G}^{i}$\}。该过程首先涉及从第 $i$ 个文件中提取第 $j$ 个函数 $f_{j}^{i}$,并构建一个提示符 $p_{s}||f_{j}^{i}$,该提示符包含转换策略 $s$、提取的函数 $f_{j}^{i}$ 以及相关上下文信息(如全局变量和头文件)。然后我们得到转换后的函数 $\hat{f_{j}^{i}} = LLM(p_{s}||f_{j}^{i})$ 。随后,将修改后的函数 $f_{j}^{i}$ 合并回源代码文件 $i$,生成一个修改后的文件,其中函数 \{$\hat{f_{2}^{i}}$, ..., $\hat{f_{j}^{i}}$\} 被修改,而其余函数 \{$f_{j+1}^{i}$, ..., $f_{G}^{i}$\} 保持不变。最后,重构的文件被编译以生成变种恶意软件 $\hat{M_{s}}$。

