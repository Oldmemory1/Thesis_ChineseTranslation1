\chapter{相关工作}

恶意软件变体生成的研究已探索多种方法。大量工作聚焦于全局或局部修改恶意软件的二进制代码,通过在特定位置注入或追加字节而不改变其行为来保持原始功能\parencite{Qiao2022},\parencite{Ebrahimi2020,Kreuk2018,Yuan2020,Kolosnjaji2018,Suciu2019}。另一种方法涉及二进制多样化技术以全局改变恶意软件的二进制文件\parencite{Lucas2021},\parencite{Lucas2023}。此外,利用贪婪算法、基于梯度的优化、生成模型和启发式技术添加无关函数来操控API调用,已成为重要研究领域\parencite{Digregorio2024}, \parencite{Hu2017,Kawai2019,Verwer2020}。另有方法通过修改底层汇编代码或使用搜索算法/基于学习的优化在特征空间内改变控制流图\parencite{Ling2024}, \parencite{Zhang2022}。直接扰动恶意软件代码空间的研究虽较少探索,但包括注入汇编代码以调用外部DLL来触发额外API而不改变控制流\parencite{Lu2022}。Murali等人\parencite{Ming2017}提出的方法操作LLVM生成的中间表示,通过策略性转换直接修改系统调用有向图,随后重新生成恶意软件可执行文件。Choi等人\parencite{Choi2019}采用名为AMVG的自适应框架,通过解析源代码并运用遗传算法生成恶意软件变体。他们在部分Python样本和良性C程序上展示了简单转换的结果,但仅限于复杂度较低的案例。