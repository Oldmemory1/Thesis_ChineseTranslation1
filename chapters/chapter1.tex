%%
% BIThesis 研究生学位论文模板 The BIThesis Template for Graduate Thesis
% This file has 否 copyright assigned and is placed in the Public Domain.
%%

\chapter{引言}
恶意软件(Malware)继续随着技术的快速扩张而激增。到2025年,网络犯罪造成的损失预计将达到每年10.5万亿美元~\parencite{PIA2024}。每秒大约发生19万起新的恶意软件事件~\parencite{AVG2024},而2024年勒索软件的平均赎金要求预计将达到每次攻击273万美元,较往年急剧上升~\parencite{Sophos2024}。尽管经过数十年的研究和缓解努力,这些数字突显了恶意软件研究在当今不断演变的威胁环境中的紧迫重要性。

现代最具变革性的AI技术之一是大型语言模型(LLMs),它在自然语言处理(NLP)~\parencite{Dubey2024,Kedia2021,Jiang2023}、代码生成~\parencite{Hou2023,MAI2024,Roziere2023,Zhu2024,Lozhkov2024,Huang2024}以及代码编辑和重构等软件工程任务~\parencite{Cassano2024,Guo2024,Cordeiro2024}中展现了非凡的能力。鉴于这些优势和进步,利用LLMs进行恶意软件源代码转换是自然的发展。最近一项针对全球行业1800名安全负责人的调查\parencite{Darktrace2024}发现,74\%的人正经历着显著的AI驱动的威胁,60\%的人感觉准备不足,无法抵御这些威胁。尽管当前的模型仅从文本生成功能完整的恶意软件存在显著局限性,但研究表明它们可以生成恶意行为者能够组装成可操作恶意软件的代码片段\parencite{Botacin2023}。LLM能力的进步与恶意软件威胁的演变相结合,为对手使用这些模型创建新恶意软件并将现有代码库变异成更难以捉摸和更具破坏性的变种铺平了道路。尽管恶意软件源代码比二进制文件更难获取,但能够访问源代码的对手,例如恶意软件作者、泄露存储库的用户或修改开源恶意软件的人,仍然可以利用LLMs生成新的、更难检测的变种。这些模型使攻击者能够持续精进和扩展其武器库,从而大规模增加恶意活动的持久性和规避性。

\section{先前的研究}
先前的研究提出了各种创建恶意软件变种的方法\parencite{Botacin2023,Qiao2022,Choi2019,Lucas2021,Ming2017,Ling2024,Digregorio2024}。然而,这些方法在至少以下一个方面表现出局限性(如表\ref{tab:1.1}所示)(A) 大多数现有方法没有利用LLMs来转换恶意软件的源代码\parencite{Qiao2022,Choi2019,Lucas2021,Ming2017,Ling2024,Digregorio2024};(B) 大多数方法依赖迭代算法来生成恶意软件变种\parencite{Qiao2022,Choi2019,Lucas2021}, \parencite{Ling2024}, \parencite{Digregorio2024};(C) 使用LLMs进行变种生成的方法,直接从成功率低的提示词开始\parencite{Botacin2023}。此外,尚不清楚生成的恶意软件在规避广泛使用的防病毒引擎方面是否表现更优。鉴于目前的情况,我们的工作引入了一种与现有恶意软件变种生成方法截然不同的方法。与大多数先前主要依赖基于对抗性机器学习或基于搜索的方法的研究不同,我们的方法独特地利用LLMs在源代码级别进行操作。基本上,它从恶意软件源代码开始,以高成功率和最少的手动工作生成变种。此外,我们的方法不需要迭代训练或基于搜索的优化,这使其与现有的恶意软件转换方法根本不同。因此,我们提出了一个尚未充分探索的新研究方向。

\begin{table}[htbp]
	\centering
	\caption{与先前研究的对比}
	\label{tab:1.1}
	\begin{tabular*}{\textwidth}{@{\extracolsep{\fill}}ccccc}
		\toprule
		方法 & 源代码 & LLM使用 & 无需训练或迭代 & 逃逸提升 \\
		\midrule
		{\parencite{Qiao2022}} & 否 & 否 & 否 & 是 \\
		{\parencite{Digregorio2024}} & 否 & 否 & 否 & 是 \\
		{\parencite{Lucas2021}} & 否 & 否 & 否 & 是 \\
		{\parencite{Ling2024}} & 否 & 否 & 否 & 是 \\
		{\parencite{Ming2017}} & 否 & 否 & 是 & 是 \\
		{\parencite{Choi2019}} & 是 & 否 & 否 & 是 \\
        {\parencite{Botacin2023}} & 否 & 是 & 是 & 否 \\
        LLMalMorph & 是 & 是 & 是 & 是 \\
		\bottomrule
	\end{tabular*}
\end{table}

\section{问题描述}
鉴于现有方法的局限性以及LLMs(特别是代码生成方面)的最新进展,我们旨在回答以下问题——我们能否利用预训练LLMs的生成能力,无需额外微调,来开发一个半自动化且高效的框架,以生成保留功能语义的恶意软件变种,这些变种能够规避广泛使用的防病毒引擎和机器学习分类器?

\section{我们的方法}
在本文中,我们对上述问题给出了肯定的回答。我们设计、实现并评估了LLMalMorph——一个专门用于生成用C/C++编写的Windows恶意软件功能变种的框架。我们只专注于Windows恶意软件,因为它在消费者和企业环境中广泛使用,仍然是恶意软件最常针对的操作系统\parencite{AVG2024OS}, \parencite{Statcounter2024}。

LLMalMorph结合了自动化代码转换和人工监督来生成恶意软件变种。该框架利用一个开源的LLM,应用精心设计的转换策略和提示词工程,在保持结构和功能完整性的同时,高效地修改恶意软件组件。人机协同(human-in-the-loop)过程处理复杂转换和多文件恶意软件中的错误,允许进行调试和配置调整。这种半自动化方法也使我们能够量化基于LLM从源代码生成恶意软件变种中的人力投入。

\section{实验和分析}
我们选择了10个不同复杂度的恶意软件样本,使用6种代码转换策略结合一个LLM生成了618个变种。我们使用主要依赖基于签名的检测和静态分析的引擎的VirusTotal\footnote{https://www.VirusTotal.com/gui/home}和Hybrid Analysis\footnote{https://hybrid-analysis.com/}评估了防病毒(AV)检测率,并测试了语义保留性。代码优化(Code Optimization)策略在两种工具上均持续实现了较低的检测率。平均而言,相对于每个样本的基准检测率,LLMalMorph在VirusTotal上将简单样本的检测率降低了31\%,将三个更复杂样本的检测率降低了10\%至15\%;在Hybrid Analysis上,与各自基准相比,四个样本的检测率降低了8\%至13\%。除了AV工具外,我们还在一个基于机器学习(ML)的恶意软件分类器上评估了LLMalMorph,并观察到在特定样本上,优化(Optimization)策略和安全(Security)策略取得了较高的攻击成功率(分别高达89\%和91\%)。诸如优化、安全和Windows API修改等策略需要更多手动编辑,其中Windows和安全策略需要更高的调试投入。值得注意的是,四个样本中超过66\%的规避型变种保留了其语义,这证明了LLMalMorph生成功能规避型恶意软件的能力。

\section{贡献}
总结而言,我们有以下贡献:

• 我们设计并实现了LLMalMorph,一个实用的Windows恶意软件变种生成框架,它使用一个开源的LLM和基于提示词(prompt-based)的代码转换。

• 我们在LLMalMorph中设计了一个人机协同(human-in-the-loop)机制,以解决LLM在调试多文件恶意软件源代码和项目级配置方面的局限性。

• 我们进行了广泛的实验,从10个样本生成了618个恶意软件变种,并评估了它们在VirusTotal和Hybrid Analysis上的检测率和语义保留性,以及在一个机器学习分类器(ML Classifier)上的攻击成功率。

• 我们使用代码编辑工作量(code editing workload)比较了不同代码转换策略的有效性,并讨论了LLM所犯错误的类型。

\section{开源}
LLMalMorph框架及其所有相关组件可在Github\footnote{https://github.com/AJAkil/LLMalMorph}找到。