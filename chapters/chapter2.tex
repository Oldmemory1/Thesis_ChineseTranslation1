%%
% BIThesis 研究生学位论文模板 The BIThesis Template for Graduate Thesis
% This file has no copyright assigned and is placed in the Public Domain.
%%

\chapter{背景}
本节描述与恶意软件(malware)、其检测系统以及大型语言模型(LLMs)相关的各种预备知识。

\section{恶意软件和检测手段}
恶意软件(Malware)指的是对手或攻击者用来在用户不知情的情况下,未经授权访问数字设备以破坏或窃取敏感信息的恶意程序 [26]。它是一个统称(umbrella term),用于描述广泛的威胁,包括木马(Trojans)、后门(backdoors)、病毒(viruses)、勒索软件(ransomware)、间谍软件(spyware)和僵尸程序(bots)[27],针对多种操作系统,如Windows、macOS、Linux和Android,以及各种文件格式,如可移植可执行文件(Portable Executable, PE)、MachO、ELF、APK和PDF [28]。在入侵系统后,恶意软件可以执行各种恶意活动,例如渗透网络、加密数据以勒索赎金或降低系统性能。

检测引擎和工具采用各种方法和工具来检测恶意软件。它们可以大致分为静态(static)、动态(dynamic)和混合(hybrid)方法 [28], [29]。静态检测在不执行恶意软件的情况下对其进行分析,依赖于诸如PE头信息(PE header information)、可读字符串(readable strings)和字节序列(byte sequences)等特征 [28]。动态检测涉及在受控环境(例如沙箱,sandboxes)中执行恶意软件,以监视运行时行为,如注册表修改(registry modifications)、进程创建(process creation)和网络活动(network activity)[28], [29]。混合检测结合了静态和动态特征,使用诸如操作码(opcodes)、对系统的API调用(API calls to the system)和控制流图(control flow graphs, CFGs)等数据 [28]。此外,基于启发式的检测(heuristic-based detection)使用启发式规则(heuristic rules)静态分析代码并动态分析行为,以确定恶意性 [30]。
