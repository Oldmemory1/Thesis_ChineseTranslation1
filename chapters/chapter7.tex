\chapter{结论和未来工作}
本研究中,我们提出了LLMalMorph框架——该框架利用LLM通过工程化提示和代码转换策略生成恶意软件变体。采用6种策略生成618个变体,证明了特定转换能降低AV检测率,并在机器学习分类器上具有显著攻击成功率。同时观察到复杂恶意软件常需大量调试以维持功能,这凸显了人工监督的必要性、提示设计的谨慎性,以及当前LLM在恶意软件源代码转换中的局限性。

尽管本研究展示了LLM生成规避性恶意软件变体的潜力,仍存在若干局限。我们的实现可通过增强提取器与合并器子模块扩展到其他语言。未来计划拓展至LLM二进制恶意软件转换,并利用LLM代理提升自动化水平。此外,旨在通过分离恶意软件相关模式改进函数选择机制,并在API序列之外开发更鲁棒的语义保留评估指标。