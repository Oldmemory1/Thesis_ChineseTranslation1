%%
% BIThesis 研究生学位论文模板 The BIThesis Template for Graduate Thesis
% This file has no copyright assigned and is placed in the Public Domain.
% Compile with: xelatex -> biber -> xelatex -> xelatex
%%

% 请勿删除下面两行注释,以免影响编译。
% !TeX program = xelatex
% !BIB program = biber

% 硕士论文模板 type=master
% 博士论文模板 type=doctor
% 开启盲审格式 blindPeerReview=true (如:[type=master,blindPeerReview=true])
% 开启双面打印模式 twoside=true (如:[type=master,twoside=true])
%     在双面打印模式下,需要放在奇数页的页面后会自动插入一个空白页,以方便直接在打印机上“双面打印”。
%
% 在 Linux 和 macOS 系统下,LaTeX 发行版默认使用的中文字体和 Windows 系统下的字体不同。
% 如果想要获得与 Word 文档相同的效果,请参阅在线帮助 https://bithesis.bitnp.net/faq/word-font.html
%
% **更多使用说明请参考 bithesis.pdf **

\documentclass[type=master,twoside=false]{bithesis}

% 此处仅列出常用的配置。全部配置用法请见「bithesis.pdf」手册。
\BITSetup{
  cover = {
    %% 使用以下参数来自定义封面日期
    date = 2022年6月,
    autoWidthPadding = 0.25em,
  },
  info = {
    % 想要删除某项封面信息,直接删除该项即可。
    % 想要让某项封面信息留空(但是保留下划线),请传入空白符组成的字符串,如"{~}"。
    % 如需要换行,则用 “\\” 符号分割。
    classification = TQ028.1,
    UDC = 540,
    title = 形状记忆聚氨酯的合成及其在织物中的应用,
    % 如需覆盖竖排标题,请配置以下选项。
    % 下面的例子展示了如何在竖排标题中使用垂直或者旋转的英文。
    % verticalTitle = {形状记忆聚氨酯{L } {T } {X }的合成 \rotatebox[origin=c]{-90}{Feng Kaiyu} 及其在织物中的应用},
    titleEn = {Synthesis and Application on textile of the Shape Memory Polyurethane},
    author = 张三,
    authorEn = San Zhang,
    studentId = 31xxxxxxxx,
    school = 材料学院,
    schoolEn = Materials Science and Engineering,
    supervisor = 李四教授,  % 指导教师
    supervisorEn = Prof. Si Li,
    chairman = 王五教授,  % 答辩委员会主席
    chairmanEn = Prof. Wang Wu,
    % -------------
    % 请按自身研究生类型填写以下项目
    %
    % --- 学术型 ---
    % degreeType = academic,
    % degree = 工学博士,  % 申请学位
    % degreeEn = Doctor of Engineering,
    % major = 材料科学与工程,  % 一级学科
    % majorEn = Materials Science and Engineering,
    %
    % --- 专业型 ---
    degreeType = professional,
    industrialMentor = 李五教授,  % 行业合作导师
    industrialMentorEn = Prof. Wu Li,
    degree = 材料与化工博士,  % 申请类别
    degreeEn = {Doctor of Materials \\ and Chemical Engineering},
    major = 材料工程,  % 学位领域
    majorEn = Materials Science and Engineering,
    % -------------
    %
    % 如果想要手动控制盲审模式下的隐藏信息,可以使用宏 \SecretInfo{}。使用方式有两种,如:
    % major = \SecretInfo{材料科学与工程} 可以得到 ******* (用等量的替换符号替代)
    % major = \SecretInfo{材料科学与工程}[ABCDEF] 可以得到 ABCDEF (用你自定义的内容替代)
    %
    defenseDate = 2025年6月,
    defenseDateEn = {June, 2025},
    keywords = {形状记忆;聚氨酯;织物;合成;应用\textcolor{blue}{(硕士一般选3~6个单词或专业术语,博士一般选3~8个单词或专业术语,且中英文关键词必须对应。【提醒】关键词不要与学科或专业名称等专有名词一致或相近。)}——请在“main.tex”开头设置},
    keywordsEn = shape memory properties; polyurethane; textile; synthesis; application,
    %
    % 必要时置于封面右上角,并按照国家规定进行标记。
    % classifiedLevel = 密级\BigStar 保密期限,
    %
    % 特别类型——工程硕博士专项
    % 工程硕博士专项 = true,
    % 特别类型——交叉研究方向(一般不用勾选)
    % crossResearch = true,
    % 特别类型——政府项目留学生(一般不用勾选)
    % internationalStudentUGP = true,
    % 以上三个选项不勾选时将会隐藏显示。
  },
  % 在目录页中不显示摘要和主要符号对照表的标题。
  TOC = {
    abstract = false,
    abstractEn = false,
    symbols = false,
  },
  style = {
    pageVerticalAlign = top,
    % 开启 Windows 平台下的中易宋体伪粗体。
    % windowsSimSunFakeBold = true,
    % 开启该选项后,将用 Times New Roman 的开源字体 TeX Gyre Termes 作为正文字体。
    % 这个选项适用于以下情况:
    % 1. 不想在系统中安装 Times New Roman。
    % 2. 在 Linux/macOS 下遇到 `\textsc` 无法正常显示的问题。
    % betterTimesNewRoman = true,
  },
  publications = {
    % 以下两个选项将影响「攻读学位期间发表论文与研究成果清单」中名称列表的省略阈值。
    % 一般来说,如果你在全部文献中最低排在第四位,建议你将两个值都设置为大于等于 4 的值。
    % 更详细的说明请见手册。
    maxbibnames = 10,
    minbibnames = 10,
    % 「攻读学位期间发表论文与研究成果清单」默认按学校要求,“按发表的时间顺序列出”。
    % 如需调整,可修改以下选项,详见 https://bithesis.bitnp.net/faq/bib-sort.html
    % sorting = false
  },
  % 采用章节标题级别的附录格式
  appendices / chapterLevel = true,
  const = {
    % 关于题名页的字段名称,截至2025年三月末,以下几项规定不完全一致:
    %     Word 模板(学位论文模版-{学术型,专业型}-2025.doc)、研函〔2018〕60号《北京理工大学研究生学位论文撰写规范》、GB/T 7713.1—2006《学位论文编写规则》
    % 目前的默认值按照 Word 模板,可能需要手动调整。
    % 例如取消注释下一行,会将「申请学位/类别」改为「申请学位级别」。
    % info / degree = {申\hspace{0.45ex}请\hspace{0.45ex}学\hspace{0.45ex}位\hspace{0.45ex}级\hspace{0.45ex}别},
    % 详见 https://bithesis.bitnp.net/faq/edit-const.html
  },
  misc = {
    % 关闭后,链接会用多种颜色表示,便于检查。
    % 无论是否开启,都不会影响打印效果。
    hideLinks = true,
    % 微调表格行间距
    tabularRowSeparation = 1.6,
  }
}

% 大部分关于参考文献样式的修改,都可以通过此处的选项进行配置。
% 详情请搜索「biblatex-gb7714-2015 文档」进行阅读。
\usepackage[
  defernumbers=true,
  backend=biber,
  style=gb7714-2015,
  gbalign=gb7714-2015,
  gbnamefmt=lowercase,
  gbpub=false,
  gbannote=true,
  gbpunctin=false,
  doi=false,
  url=false,
  eprint=false,
  isbn=false,
]{biblatex}

% 添加参考文献
\addbibresource{reference/main.bib}
% 攻读学位期间发表论文与研究成果清单,详细使用方法见 `chapters/pub.tex`。
\addbibresource{reference/pub.bib}


\usepackage{graphicx}

\begin{document}

% 封面绘制
\MakeCover

% 打印书脊
\MakePaperBack

% 中文信息与英文信息
\MakeTitle

% 论文原创性声明和使用授权
\MakeOriginality

%%%%%%%%%%%%%%%%%%%%%%%%%%%%%%
%% 前置部分
%%%%%%%%%%%%%%%%%%%%%%%%%%%%%%
\frontmatter

% 摘要
%%
% BIThesis 研究生学位论文模板 The BIThesis Template for Graduate Thesis
% This file has no copyright assigned and is placed in the Public Domain.
%%

\begin{abstract}
Large Language Models (LLMs) have transformed software development and automated code generation. Motivated by these advancements, this paper explores the feasibility of LLMs in modifying malware source code to generate variants. We introduce LLMalMorph, a semi-automated framework that leverages semantical and syntactical code comprehension by LLMs to generate new malware variants. LLMalMorph extracts function level information from the malware source code and employs custom-engineered prompts coupled with strategically defined code transformations to guide the LLM in generating variants without resource-intensive fine-tuning. To evaluate LLMalMorph, we collected 10 diverse Windows malware samples of varying types, complexity and functionality and generated 618 variants. Our thorough experiments demonstrate that it is possible to reduce the detection rates of antivirus engines of these malware variants to some extent while preserving malware functionalities. In addition, despite not optimizing against any Machine Learning(ML)-based malware detectors, several variants also achieved notable attack success rates against an ML-based malware classifier. We also discuss the limitations of current LLM capabilities in generating malware variants from source code and assess where this emerging technology stands in the broader context of malware variant generation.
\end{abstract}

\begin{abstractEn}
大型语言模型(LLMs)已经改变了软件开发和自动化代码生成。受这些进展的激励,本文探索了LLMs修改恶意软件源代码以生成变种的可行性。我们引入了LLMalMorph,这是一个半自动化框架,它利用LLMs对代码的语义和语法理解来生成新的恶意软件变种。LLMalMorph从恶意软件源代码中提取函数级信息,并采用定制设计的提示词结合策略性定义的代码转换,来指导LLM生成变种,而无需资源密集型的微调。为了评估LLMalMorph,我们收集了10个不同类型、复杂度和功能的多样化Windows恶意软件样本,并生成了618个变种。我们详尽的实验表明,在保持恶意软件功能的同时,可以在一定程度上降低这些恶意软件变种对防病毒引擎的检测率。此外,尽管没有针对任何基于机器学习(ML)的恶意软件检测器进行优化,一些变种在对抗一个基于ML的恶意软件分类器时也取得了显著的攻击成功率。我们还讨论了当前LLM在从源代码生成恶意软件变种方面的能力局限性,并评估了这项新兴技术在更广泛的恶意软件变种生成背景下的现状。
\end{abstractEn}


% 制作目录
\MakeTOC

% 插图目录
\listoffigures
% 表格目录
\listoftables

% 主要符号对照表
%%
% BIThesis 研究生学位论文模板 The BIThesis Template for Graduate Thesis
% This file has no copyright assigned and is placed in the Public Domain.
%%

\begin{symbols}
  \item[BIT] 北京理工大学的英文缩写
  \item[\LaTeX] 一个很棒的排版系统
  \item[\LaTeXe] 一个很棒的排版系统的最新稳定版
  \item[ctex] 成套的中文\LaTeX{}解决方案,由一帮天才们开发
  \item[$ e^{\pi{}i}+1=0$] 一个集自然界五大常数一体的炫酷方程
\end{symbols}


\mainmatter

% 请根据论文内容,按照顺序添加章节。
%%
% BIThesis 研究生学位论文模板 The BIThesis Template for Graduate Thesis
% This file has 否 copyright assigned and is placed in the Public Domain.
%%

\chapter{引言}
恶意软件(Malware)继续随着技术的快速扩张而激增。到2025年,网络犯罪造成的损失预计将达到每年10.5万亿美元~\parencite{PIA2024}。每秒大约发生19万起新的恶意软件事件~\parencite{AVG2024},而2024年勒索软件的平均赎金要求预计将达到每次攻击273万美元,较往年急剧上升~\parencite{Sophos2024}。尽管经过数十年的研究和缓解努力,这些数字突显了恶意软件研究在当今不断演变的威胁环境中的紧迫重要性。

现代最具变革性的AI技术之一是大型语言模型(LLMs),它在自然语言处理(NLP)~\parencite{Dubey2024,Kedia2021,Jiang2023}、代码生成~\parencite{Hou2023,MAI2024,Roziere2023,Zhu2024,Lozhkov2024,Huang2024}以及代码编辑和重构等软件工程任务~\parencite{Cassano2024,Guo2024,Cordeiro2024}中展现了非凡的能力。鉴于这些优势和进步,利用LLMs进行恶意软件源代码转换是自然的发展。最近一项针对全球行业1800名安全负责人的调查\parencite{Darktrace2024}发现,74\%的人正经历着显著的AI驱动的威胁,60\%的人感觉准备不足,无法抵御这些威胁。尽管当前的模型仅从文本生成功能完整的恶意软件存在显著局限性,但研究表明它们可以生成恶意行为者能够组装成可操作恶意软件的代码片段\parencite{Botacin2023}。LLM能力的进步与恶意软件威胁的演变相结合,为对手使用这些模型创建新恶意软件并将现有代码库变异成更难以捉摸和更具破坏性的变种铺平了道路。尽管恶意软件源代码比二进制文件更难获取,但能够访问源代码的对手,例如恶意软件作者、泄露存储库的用户或修改开源恶意软件的人,仍然可以利用LLMs生成新的、更难检测的变种。这些模型使攻击者能够持续精进和扩展其武器库,从而大规模增加恶意活动的持久性和规避性。

\section{先前的研究}
先前的研究提出了各种创建恶意软件变种的方法\parencite{Botacin2023,Qiao2022,Choi2019,Lucas2021,Ming2017,Ling2024,Digregorio2024}。然而,这些方法在至少以下一个方面表现出局限性(如表\ref{tab:1.1}所示)(A) 大多数现有方法没有利用LLMs来转换恶意软件的源代码\parencite{Qiao2022,Choi2019,Lucas2021,Ming2017,Ling2024,Digregorio2024};(B) 大多数方法依赖迭代算法来生成恶意软件变种\parencite{Qiao2022,Choi2019,Lucas2021}, \parencite{Ling2024}, \parencite{Digregorio2024};(C) 使用LLMs进行变种生成的方法,直接从成功率低的提示词开始\parencite{Botacin2023}。此外,尚不清楚生成的恶意软件在规避广泛使用的防病毒引擎方面是否表现更优。鉴于目前的情况,我们的工作引入了一种与现有恶意软件变种生成方法截然不同的方法。与大多数先前主要依赖基于对抗性机器学习或基于搜索的方法的研究不同,我们的方法独特地利用LLMs在源代码级别进行操作。基本上,它从恶意软件源代码开始,以高成功率和最少的手动工作生成变种。此外,我们的方法不需要迭代训练或基于搜索的优化,这使其与现有的恶意软件转换方法根本不同。因此,我们提出了一个尚未充分探索的新研究方向。

\begin{table}[htbp]
	\centering
	\caption{与先前研究的对比}
	\label{tab:1.1}
	\begin{tabular*}{\textwidth}{@{\extracolsep{\fill}}ccccc}
		\toprule
		方法 & 源代码 & LLM使用 & 无需训练或迭代 & 逃逸提升 \\
		\midrule
		{\parencite{Qiao2022}} & 否 & 否 & 否 & 是 \\
		{\parencite{Digregorio2024}} & 否 & 否 & 否 & 是 \\
		{\parencite{Lucas2021}} & 否 & 否 & 否 & 是 \\
		{\parencite{Ling2024}} & 否 & 否 & 否 & 是 \\
		{\parencite{Ming2017}} & 否 & 否 & 是 & 是 \\
		{\parencite{Choi2019}} & 是 & 否 & 否 & 是 \\
        {\parencite{Botacin2023}} & 否 & 是 & 是 & 否 \\
        LLMalMorph & 是 & 是 & 是 & 是 \\
		\bottomrule
	\end{tabular*}
\end{table}

\section{问题描述}
鉴于现有方法的局限性以及LLMs(特别是代码生成方面)的最新进展,我们旨在回答以下问题——我们能否利用预训练LLMs的生成能力,无需额外微调,来开发一个半自动化且高效的框架,以生成保留功能语义的恶意软件变种,这些变种能够规避广泛使用的防病毒引擎和机器学习分类器?

\section{我们的方法}
在本文中,我们对上述问题给出了肯定的回答。我们设计、实现并评估了LLMalMorph——一个专门用于生成用C/C++编写的Windows恶意软件功能变种的框架。我们只专注于Windows恶意软件,因为它在消费者和企业环境中广泛使用,仍然是恶意软件最常针对的操作系统\parencite{AVG2024OS}, \parencite{Statcounter2024}。

LLMalMorph结合了自动化代码转换和人工监督来生成恶意软件变种。该框架利用一个开源的LLM,应用精心设计的转换策略和提示词工程,在保持结构和功能完整性的同时,高效地修改恶意软件组件。人机协同(human-in-the-loop)过程处理复杂转换和多文件恶意软件中的错误,允许进行调试和配置调整。这种半自动化方法也使我们能够量化基于LLM从源代码生成恶意软件变种中的人力投入。

\section{实验和分析}
我们选择了10个不同复杂度的恶意软件样本,使用6种代码转换策略结合一个LLM生成了618个变种。我们使用主要依赖基于签名的检测和静态分析的引擎的VirusTotal\footnote{https://www.VirusTotal.com/gui/home}和Hybrid Analysis\footnote{https://hybrid-analysis.com/}评估了防病毒(AV)检测率,并测试了语义保留性。代码优化(Code Optimization)策略在两种工具上均持续实现了较低的检测率。平均而言,相对于每个样本的基准检测率,LLMalMorph在VirusTotal上将简单样本的检测率降低了31\%,将三个更复杂样本的检测率降低了10\%至15\%;在Hybrid Analysis上,与各自基准相比,四个样本的检测率降低了8\%至13\%。除了AV工具外,我们还在一个基于机器学习(ML)的恶意软件分类器上评估了LLMalMorph,并观察到在特定样本上,优化(Optimization)策略和安全(Security)策略取得了较高的攻击成功率(分别高达89\%和91\%)。诸如优化、安全和Windows API修改等策略需要更多手动编辑,其中Windows和安全策略需要更高的调试投入。值得注意的是,四个样本中超过66\%的规避型变种保留了其语义,这证明了LLMalMorph生成功能规避型恶意软件的能力。

\section{贡献}
总结而言,我们有以下贡献:

• 我们设计并实现了LLMalMorph,一个实用的Windows恶意软件变种生成框架,它使用一个开源的LLM和基于提示词(prompt-based)的代码转换。

• 我们在LLMalMorph中设计了一个人机协同(human-in-the-loop)机制,以解决LLM在调试多文件恶意软件源代码和项目级配置方面的局限性。

• 我们进行了广泛的实验,从10个样本生成了618个恶意软件变种,并评估了它们在VirusTotal和Hybrid Analysis上的检测率和语义保留性,以及在一个机器学习分类器(ML Classifier)上的攻击成功率。

• 我们使用代码编辑工作量(code editing workload)比较了不同代码转换策略的有效性,并讨论了LLM所犯错误的类型。

\section{开源}
LLMalMorph框架及其所有相关组件可在Github\footnote{https://github.com/AJAkil/LLMalMorph}找到。
%%
% BIThesis 研究生学位论文模板 The BIThesis Template for Graduate Thesis
% This file has no copyright assigned and is placed in the Public Domain.
%%

\chapter{具体研究内容}

具体研究内容是学位论文的主要部分,是研究结果及其依据的具体表述,是研究能力的集中体现,一般应包括第2章、第3章至结论前一章。具体研究内容应该结构合理,层次清楚,重点突出,文字简练、通顺。可包括以下各方面:研究对象、研究方法、仪器设备、材料原料、实验和观测结果、理论推导、计算方法和编程原理、数据资料和经过加工整理的图表、理论分析、形成的论点和导出的结论等。具体研究内容各章后可有一节“本章小结”(必要时)。

\begin{them}[留数定理]
\label{thm:res}
  假设$U$是复平面上的一个单连通开子集,$a_1,\ldots,a_n$是复平面上有限个点,$f$是定义在$U\backslash \{a_1,\ldots,a_n\}$上的全纯函数,
  如果$\gamma$是一条把$a_1,\ldots,a_n$包围起来的可求长曲线,但不经过任何一个$a_k$,并且其起点与终点重合,那么:
  \begin{equation}
    \label{eq:res}
    \ointop_{\gamma}f(z)\,\mathrm{d}z = 2\pi\mathbf{i}\sum^n_{k=1}\mathrm{I}(\gamma,a_k)\mathrm{Res}(f,a_k)
  \end{equation}

  如果$\gamma$是若尔当曲线,那么$\mathrm{I}(\gamma, a_k)=1$,因此:
  \begin{equation}
    \label{eq:resthm}
    \ointop_{\gamma}f(z)\,\mathrm{d}z = 2\pi\mathbf{i}\sum^n_{k=1}\mathrm{Res}(f,a_k)
  \end{equation}

  在这里,$\mathrm{Res}(f, a_k)$表示$f$在点$a_k$的留数,$\mathrm{I}(\gamma,a_k)$表示$\gamma$关于点$a_k$的卷绕数。
  卷绕数是一个整数,它描述了曲线$\gamma$绕过点$a_k$的次数。如果$\gamma$依逆时针方向绕着$a_k$移动,卷绕数就是一个正数,
  如果$\gamma$根本不绕过$a_k$,卷绕数就是零。
\end{them}

\begin{proof}
  首先,由……

  其次,……

  所以,由\autoref{thm:res}可知……
  \qedhere
\end{proof}

\textit{有关公式与上下文间距的一些注意事项:请保证源码中的公式的环境(如}
\\ \verb|\begin{equation}|
  \textit{)与上一段落不要有空行。否则,公式和上文段落之间会有额外的空白。}


\backmatter

% 结论
%%
% BIThesis 研究生学位论文模板 The BIThesis Template for Graduate Thesis
% This file has no copyright assigned and is placed in the Public Domain.
%%

\begin{conclusion}

本文采用……。{\color{blue}(结论作为学位论文正文的最后部分单独排写,但不加章号。结论是对整个论文主要结果的总结。在结论中应明确指出本研究的创新点,对其应用前景和社会、经济价值等加以预测和评价,并指出今后进一步在本研究方向进行研究工作的展望与设想。结论部分的撰写应简明扼要,突出创新性。)}

\end{conclusion}

% 参考文献
%%
% BIThesis 研究生学位论文模板 The BIThesis Template for Graduate Thesis
% This file has no copyright assigned and is placed in the Public Domain.
%%

%
% 如无特殊需要,本页面无需更改。
%
% **注意:如果发现渲染出来的文献编号不正确,请同时使用以下两个方式解决:**
% 1. 清除缓存后重新编译(比如使用 `latexmk -c`)。
% 2. 请确保无编译错误。


\begin{bibprint}
  \printbibliography[heading=none,notcategory=mypub,resetnumbers=true]
\end{bibprint}


% 附录
%%
% BIThesis 研究生学位论文模板 The BIThesis Template for Graduate Thesis
% This file has no copyright assigned and is placed in the Public Domain.
%%

\begin{appendices}
\chapter{不同子程序的细节}
\section{A. 关于基于LLM的代码生成的细节}
如算法1所述,我们将提示连同函数和相关信息传递给LLM,以生成修改后的代码。LLM处理该提示并产生响应,我们解析该响应以提取生成的代码。在生成代码时,我们遇到了两个关键挑战性案例:

• LLM偶尔未能以期望的格式生成代码。
• LLM有时会描述给定函数的机制,而不是生成任何代码。

在第二种情况中,模型要么提供了对输入函数的详细解释,要么基于提供的上下文将该函数识别为潜在恶意并生成了对给定代码的分析。为了解决这些挑战,我们实现了一个重试机制。当任一情况发生时,我们使用不同的随机种子最多提示LLM五次。如果LLM仍然未能生成可用的代码,我们则回退到原始函数并继续进行解析代码的下一步。由于我们提示模型以特定格式生成代码,因此我们可以从生成的响应中解析代码,保存生成的代码,并记录有用的信息以供我们的指标计算,例如生成的代码行数、生成代码所需的时间、生成代码的路径等。我们在算法~\ref{alg:Malware Variant Generation}后续的合并阶段利用了这些信息。

\section{B. 关于“merger”子程序的细节}
该子程序仔细追踪哪些函数需要更新、哪些保持不变,在整个过程中保持两者之间的明确区分。它保留源文件中的原始头声明和全局变量,以保持项目范围内的一致性。这种方法防止了文件内部或延伸到其他文件中可能出现的依赖关系的潜在破坏。此外,我们指示LLM避免声明任何全局变量,并仅依靠创建和使用函数内部的局部变量。这确保了模块化并防止了意外的副作用。在转换后的函数被拆分为子函数的情况下,“merger”子程序有条不紊地定义它们的原型,将这些子函数放置在文件的开头,并整合了修改后的原始函数,该函数现在调用这些子函数。
通过遵循这些准则,“merger”子程序确保了转换后函数的集成是健壮和结构化的,并最大程度地降低了在代码库中引入不一致性的风险。


\chapter{详细的恶意软件描述}
我们提供了为进行实验而选择的每个恶意软件样本的详细描述。我们还通过Triage Sandbox运行了所有样本,以从沙箱报告中了解它们的行为。对于两个样本,沙箱未提供有用信息,但我们添加了其余样本的描述,我们从沙箱中获得这些样本的信息。

Exeinfector

Exeinfector在相关的GitHub仓库[38]中被归类为感染型,并且VirusTotal为其标记了诸如持久化、长时间休眠、反调试(以及用户输入检测等行为。Triage沙箱报告显示了恶意活动,包括添加持久化运行项、修改注册表、在system32目录中释放文件以及执行系统语言发现。

Fungus

Fungus在相关的GitHub仓库[38]中被归类为通用犯罪软件,是一种复杂的多文件C++恶意软件。VirusTotal将其与ircbot和autorun7等家族标签相关联。它具有反沙箱技术、基于USB的传播、服务器通信、防火墙规避以及键盘记录能力。Triage报告指出其活动包括设置自动启动、加载DLL、执行释放的文件、修改注册表、执行系统语言和位置发现以及可疑地使用Windows API调用。

Dexter
Dexter是一种销售点木马,被识别为针对运行Microsoft Windows的POS机的恶意软件[55]。它于2012年被发现,以从POS机窃取信用卡和借记卡信息、将收集的数据发送到预定服务器以及表现出类僵尸程序行为而闻名[56]。VirusTotal将Dexter分类为木马、勒索软件和下载器,带有poxters和dexter等家族标签。Triage报告指出它会删除自身、崩溃、通过修改注册表添加运行项以实现持久化、执行系统语言发现以及使用可疑的Windows API函数,如“AdjustPrivilegeToken”、“WriteProcessMemory”和“EnumerateProcess”

HiddenVNC Bot

HiddenVNC,根据其源代码readme文件开发于2021年,是一种多文件的复杂C++恶意软件。VirusTotal将其分类为木马和银行木马,将其与tinukebot、tinynuke和tinuke等家族标签相关联,表明它可能是以开后门、窃取信息和下载恶意文件而闻名的Tinynuke木马僵尸程序的一个实现[57]。该readme将其描述为一个隐藏虚拟网络计算工具,它创建一个“隐藏桌面”,允许攻击者在用户不知情的情况下控制目标机器。其功能包括控制多台机器、启动应用程序(例如浏览器、PowerShell)以及在隐藏桌面上执行远程命令。它生成两个可执行文件;我们使用VirusTotal检测率为76.503\%的Client.exe,因为它比Server.exe(15.277\%)更具恶意性。

Predator

Predator,也称为Predator the Thief,于2018年中首次被发现[58]。该样本是一种复杂的C++信息窃取木马,从其受感染主机窃取广泛的数据,例如系统信息、存储的浏览器密码、cookie、表单数据,甚至加密货币钱包地址[58]。它还能捕获网络摄像头照片、记录击键、从应用程序(例如VPN、FTP、游戏客户端)提取凭证,并收集剪贴板内容和加密货币钱包文件[59]。VirusTotal将其归类为木马,并与窃取器、adwarex和fragtor等家族标签相关联。Triage沙箱报告向我们显示,它读取FTP客户端存储的数据文件、读取Web浏览器的用户/配置文件数据、从不安全的文件中窃取凭证、访问加密货币钱包并可能执行凭据收集。

Prosto

Prosto,也称为ProstoStealer,同样是一种信息窃取木马,但比Predator样本大得多且复杂得多,并且是用C++编写的。它利用受害机器窃取有价值且有用的信息,如登录信息、凭证、密码和直接文件。所有细节都存储在由攻击者控制的服务器中,以便后续用于诈骗和欺诈活动[60]。VirusTotal将此样本分类为木马、病毒和间谍软件,并用fragtor和convagent等家族标签进行标记。该样本的Triage沙箱报告告诉我们,它会检查计算机位置设置、读取Web浏览器的用户数据、修改Internet Explorer设置,并且还有可疑地使用“FindShellTrayWindow”、“SetWindowsHookEx”、“WriteProcessMemory”等方法。

Conti

Conti勒索软件于2019年末出现[61]。这是一种极其复杂的恶意软件,包含不同的活动部件,拥有超过8000行用C++编写的代码。在我们的实验中,我们使用该样本的cryptor可执行文件。它采用双重勒索策略,在窃取数据的同时加密文件,以迫使受害者支付赎金。它以其快速的加密速度和针对医疗保健等关键行业而闻名[61]。VirusTotal将该可执行文件分类为木马和勒索软件,并将其与conti和adwarex等家族标签相关联。

Babuk

Babuk,也称为Babyk,是一种复杂的勒索软件,于2021年初被发现[62],针对多个平台,如Windows、适用于Linux的ARM以及VMware ESXI环境,并使用椭圆曲线算法构建加密密钥[63]。该勒索软件用C++编写,约4000行代码,极其复杂,针对多个国家和地区的医疗保健、塑料、运输、电子和农业等行业[62]。我们在实验中使用了该勒索软件的加密模块,并使用了生成的“.bin”格式可执行文件。VirusTotal将其归类为勒索软件和木马,并用babuk、babyk和epack等家族标签进行标记。Triage报告向我们显示,它是一种babuk加密器并属于babuk家族,会删除自身的卷影副本、重命名多个文件并添加文件扩展名、枚举已连接的驱动器和物理存储设备、与卷影副本交互,并且还有各种可疑的Windows API使用。

RedPetya Ransomware

RedPetya属于Petya家族加密恶意软件,于2016年首次被发现[64]。它采用引导锁式加密,在感染受害者后覆盖系统的主引导记录并强制重启;此时不是加载Windows,而是显示一个虚假屏幕,同时恶意软件使用加密算法在磁盘上秘密加密NTFS主文件表[65]。我们使用了其源代码的一个开源版本,该版本用C++编写,约1500行代码,使用OpenSSL进行加密,是对RedPetya恶意软件的完全重写。VirusTotal将此样本分类为木马和勒索软件,并给出petya、heur3和diskcoder等家族标签。Triage报告还显示它具有持久性,是一个引导区病毒,并且会写入主引导记录。它还显示可疑地使用Windows API,如“EnumeratesProcesses”、“AdjustPrivilegeToken”。

RansomWar

该样本是一种相对简单的勒索软件,用C语言编写,共1377行代码,使用河豚加密算法加密文件,并且代码中内置了邮件发送机制。VirusTotal将其归类为木马、勒索软件,并给出barys、ransomware等家族标签。从Triage沙箱中,我们得知它会枚举已连接的驱动器并在System32目录中释放文件。



\chapter{用于LLM修改的文件及恶意软件函数选择数量}

本节阐述我们从每个恶意软件样本中选择待修改函数数量的方法。表~\ref{tab:4.1}虽概述了各恶意软件样本的函数总量,但我们的选择过程需经严格筛选以确保修改的有效性。我们排除了属于外部库(如加密库和头文件)的文件,仅聚焦于包含自定义恶意软件代码的文件,并仅考虑按算法~\ref{alg:Function Transformation Using LLM}顺序修改全局函数。

鉴于不同恶意软件样本的函数数量存在差异,我们采用系统化策略:对函数较少的样本修改更大比例,而对函数数量显著的样本则修改较小比例。该方法在保证充分修改覆盖率与LLMalMorph所需可控人工调试工作量之间取得了平衡。

应用该方法,我们选取并修改了如下函数:
\begin{table}[htb]
	\centering
	\caption{修改的函数选择标准}
	\label{tab:function_selection_criteria_for_modification}
    \begin{tabularx}{\textwidth}{@{} 
    >{\RaggedRight}p{0.5\textwidth} 
    >{\RaggedRight}p{0.5\textwidth} 
    @{}}
		\toprule
		函数数量 & 修改率 \\
		\midrule
		\textless 10  & 100\% \\
		10 - 20 & 60\% \\
        20 - 40  & 30\% \\
		40 - 70 & 20\% \\
        \textgreater 70  & 15\% \\
		\bottomrule
	\end{tabularx}
\end{table}

应用此方法,我们选择并修改了以下函数:
\begin{table}[htb]
	\centering
	\caption{每个恶意软件样本的函数选择}
	\label{tab:function_selection_for_each_malware_sample}
    \begin{tabularx}{\textwidth}{@{} 
    >{\RaggedRight}p{0.25\textwidth} 
    >{\centering}p{0.25\textwidth} 
    >{\centering}p{0.25\textwidth} 
    >{\RaggedRight}p{0.25\textwidth} 
    @{}}
		\toprule
		恶意软件样本 & 总共选择的函数 & 修改的函数 & 修改率 \\
		\midrule
		Exeinfector & 4 & 4 & 100\% \\
		Fungus & 46 & 9 & 20\% \\
        Dexter & 61 & 12 & 20\% \\
		HiddenVNC bot & 60 & 12 & 20\% \\
        Predator & 30 & 9 & 30\% \\
        Prostostealer & 70 & 14 & 20\% \\
		Conti ransomware & 93 & 14 & 15\% \\
        Babuk ransomware & 35 & 11 & 30\% \\
		RedPetya & 15 & 9 & 60\% \\
        ransomware & 9 & 9 & 100\% \\
		\bottomrule
	\end{tabularx}
\end{table}

在必要处我们进行向上取整。对于RansomWar样本,我们最初尝试修改包含4个函数的"blowfish.c"文件未能成功,这源于LLM即使在纠错机制下仍存在生成函数变体的限制。因此,我们将修改转向包含9个函数的"RansomWar.c"文件。由于该文件属于函数数量少于10的样本类别,我们修改了其全部函数。总体而言,这种结构化方法使我们能在保持一致性的同时,确保不会在函数数量庞大的样本(例如Conti勒索软件)中修改过多函数——这考虑到LLMalMorph所需的人工调试需求。
\chapter{额外恶意软件检测率分析}
VirusTotal的Dexter分析

如图\ref{fig:5.1}中Dexter的第三子图所示,优化策略检测率持续下降,在六次函数修改后稳定在均值附近。当修改文件"injectSection"(负责进程代码注入和资源管理)中的第二个函数后,"复用性"策略检测率骤降至67.847\%。尽管该函数初始导致检测率下降,但随后回升至74.537\%。其余策略检测率则始终接近整体平均值。

VirusTotal的Prosto分析

如图\ref{fig:5.1}中Prosto窃取器样本的第六子图所示,检测率波动显著:复用性策略在第4至第6次修改间急剧下降,优化策略则在第10至第11次修改间出现陡降。优化、Windows及安全策略均呈下降趋势,其中优化策略最低达52.738\%,较基线62.033\%降低9.295\%。LLM采用替代性Windows API函数进行Base64编码和HTTP连接管理可能导致此下降。

VirusTotal的Babuk分析

如图\ref{fig:5.1}中Babuk勒索软件样本的第八子图所示,优化策略在第二个函数修改起始处出现显著下降,检测率达64.861\%,较基线71.759\%降低近7\%。该变体在此策略的分数微升但仍低于基线检测率。除"复用性"外所有策略均呈现类似趋势,但检测率下降幅度不大。

混合分析的Predator分析

如图\ref{fig:5.2}中Predator窃取器样本的第五子图所示,除Predator子图中少量偏移数据点外,多数变体呈现极小波动。值得注意的是,优化策略检测率略低于其他变体。

混合分析的Prosto分析

如图\ref{fig:5.2}中Prosto窃取器样本的第六子图所示,多数变体聚集在72.33\%的基线检测率附近且偏差较小。未发现任何特定策略变体呈现显著低于基线的检测率。

混合分析的Conti分析

在Conti勒索软件的分析图中,71.568\%的平均检测率较基线79.333\%低约8\%。优化策略在第5至第6个函数修改处出现急剧下降,后续函数检测率维持在65\%左右。此外,质量策略的检测率同样呈现下降趋势。
\chapter{机器学习模型与阈值详情}

本节详述机器学习模型的具体架构。Malconv主要采用卷积神经网络设计,通过将恶意软件作为原始字节流处理进行分类。ResNet50分类器底层使用原始ResNet50模型\parencite{He2016},该模型首先将恶意软件转换为灰度图像,再利用图像进行分类。而Malgraph模型不直接使用图像/可执行文件,它是一种分层图恶意软件分类器,采用两个基于图神经网络(GNN)的编码层:函数内层将单个函数的控制流图(CFG)编码为向量,函数间层则利用前层生成的向量及外部函数,对函数调用图(FCG)表征进行编码以学习全局程序表示。最终预测层对此嵌入向量应用多层感知机(MLP)计算恶意概率。

对于Malconv和MalGraph,我们采用文献\parencite{Ling2024}提供的现成实现,该模型基于文献\parencite{Ling2022}所引入的数据集训练——该数据集包含210,251个Windows可执行文件(101,641个恶意软件和108,610个正常软件),涵盖848个恶意软件家族。数据集构成与模型性能的补充细节详见\parencite{Ling2024}。针对ResNet50,我们采用预训练的ImageNet模型,基于文献\parencite{Li2025}最新提出的恶意软件图像表征进行微调,该数据集包含恶意软件样本(2024年3月、4月、5月、7月和8月从MalwareBazaar\footnote{https://bazaar.abuse.ch/}收集)及对应正常软件。训练数据包含7,312个恶意软件实例和14,338个正常软件实例,恶意软件与正常软件比例为0.5:1。此训练分类器的性能在独立测试集上评估,该测试集包含9月恶意软件(同样采集自Malwarebazaar)和正常软件样本,恶意软件与正常软件比例为0.44:1(1,337个恶意软件和3,020个正常软件)。分类器在此测试集达到85\%准确率和85\% F1分数,更多细节参见[43]。

对于所有三个分类器,我们采用文献\parencite{Ling2024}设定的0.1\%误报率(FPR)阈值。恶意软件检测中需要低误报率以减少对良性文件的误判,但这也提高了决策阈值,可能导致更高的攻击成功率。此权衡关系已在先前研究\parencite{Ling2024}中被指出。针对Malconv和MalGraph,文献\parencite{Ling2024}提供的现成实现已包含0.1\% FPR阈值。对于微调后的ResNet50模型,我们通过选择良性测试集(仅用于阈值校准而非评估)上恶意软件概率的99.9百分位点计算等效阈值。在此模型特定阈值下,Malconv和ResNet50均未标记10个原始恶意软件样本,仅Malgraph标记了Fungus、Dexter、Conti和Babuk。故而我们将对抗评估集中于这四个样本变体。
\chapter{所有提示词}


我们提供通过算法~\ref{alg:Prompt Construction Subroutine for LLM-based Function Transformation}所述LLM修改LLMalMorph函数所使用的全部提示词:
\section{A. 系统提示词}
System Prompt: You are an intelligent coding assistant who is expert in writing, editing, refactoring and debugging code. You listen to exact instructions and specialize in systems programming and use of C, C++ and C\# languages with Windows platforms

\section{B. 介绍提示词}
Below this prompt you are provided headers, global variables, class and struct definitions and self.num\_functions global function definition(s) from a self.language\_name source code file. The parameters of the functions also have specific types. As an intelligent coding assistant, GENERATE one VARIANT of each of these functions: ***’, ’.join([func\_name for func\_name in self.function\_names])*** following these instructions:

\section{C. 代码转换策略提示词}
如第四-B小节所述,我们开发了六种代码转换策略。各策略具体提示词如下:

\subsection{代码优化}
1. Remove code redundancies.

2. Identify performance bottlenecks and fix them.

3. Simplify the code’s logic or structure and optimize data structures and algorithms if applicable.

4. Use language-specific features or modern libraries if applicable.

\subsection{代码质量与可靠性}
1. Check error handling and edge cases.

2. Follow coding practices and style guidelines.

3. Add proper documentation to classes and functions, and comments for complex parts.

\subsection{代码复用性}
Make the code reusable by dividing supplied functions into smaller function blocks if and where applicable. The smaller functions should be called inside the respective supplied functions as needed.

\subsection{代码安全性}
1.Identify security vulnerabilities and fix them.

2.If the function you are modifying contains cryptographic operations,change the cryptographic library used for those operations.If no cryptographic operations are present,no changes are necessary.

3.Follow secure coding standards and guidelines.

\subsection{代码混淆}
1. Change the given function’s and LOCAL variable’s names to meaningless, hard-to-understand strings which are not real words. DO NOT redefine or rename global variables (given to you) and names of functions that are called inside the given function ( might be defined elsewhere ) under any circumstances. However if the given function name is any of ‘main‘, ‘wmain‘, ‘WinMain‘, ‘wWinMain‘, ‘DllMain‘, ‘\_tWinMain‘, ‘\_tmain‘ do not change it’s name, only change the local variable’s names inside the function.

2. Add unnecessary jump instructions, loops, and conditional statements inside the functions.

3. Add unnecessary functions and call those functions inside the original functions.

4. Add anti-debugging techniques to the code.

5. If there are loops/conditional statements in the code change them to their equivalent alternatives and make them more difficult to follow.

6. Incorporate code to the variants that activates under very rare and obscure cases without altering core functionality, making the rare code hard to detect during testing.

\subsection{WindowsAPI专用转换}
1. Identify all Windows API function calls in the given functions.

2. If there are such function calls, replace each identified Windows API function call with an alternative Windows API function call or sequence of calls that achieves the same task.

3. If applicable, use indirect methods or wrappers around the Windows API calls to achieve the same functionality.

4. Ensure that the functionality remains the same after the replacement.

\section{D. 保留规则提示词}
REMEMBER, the generated code MUST MAINTAIN the same FUNCTIONALITY as the original code. Keep the usage of globally declared variables as it is. Modify ONLY the self.num\_functions free/global function(s) named ***’, ’.join([func\_name for func\_name in self.function\_names])***. If you find any custom functions/custom structure/class objects/custom types/custom variables that are used inside the given self.num\_functions function(s) but not in the provided code snippet, you can safely assume that these are defined elsewhere and you should use them in your generated code as it is. DO NOT modify the names of these and do not redefine them.

\section{E. 附加约束}
These CRUCIAL instructions below MUST ALWAYS BE FOLLOWED while generating variants:
1. You MUST NOT regenerate the extra information I provided to you such as headers, global variables, structs and classes for context.

2. If you modify the functions ***’, ’.join([func\_name for func\_name in self.function\_names])***, you MUST NOT regenerate the original code. But if a function cannot be changed, then include the original code. 

3. ONLY generate the function variants and any new headers/libraries you used.

4. You MUST NOT generate any extra natural language messages/comments.

5. You MUST Generate all the modified functions within a single ‘‘‘self.language\_name ‘‘‘ tag. For example your response should look like this for one generatedfunction named ‘int func(int a)‘: f"{example\_code}" Remember, if you have generated multiple functions, you should include all of them within the same ‘‘‘self.language\_name ‘‘‘ tag.

6. Use the global variables as they are inside your generated functions and do not change/redeclare the global variables.

7. Always complete the function that you generate. Make sure to fill up the function body with the appropriate code. DO NOT leave any function incomplete.

上述提示词使用的"example\_code":

’c’:‘‘‘self.language\_name
 \#include<stdio.h>
 intfunc(inta) \{ printf("\%d",a);returna+1; \}
‘‘‘
’cpp’:‘‘‘self.language\_name
 \#include<iostream>
 intfunc(inta) \{ cout<<a<<endl;return a +1; \}
‘‘‘
\chapter{完整提示词示例}

我们展示Fungus样本AntiSandbox()函数的完整提示词实例。此为main文件的第一个函数,亦属我们修改的第六个函数。以下示例展示优化代码转换策略的提示词。为便于理解,算法~\ref{alg:Prompt Construction Subroutine for LLM-based Function Transformation}所述用户提示词的不同部分以<<提示词名称>>符号高亮标注:

\section{A. AntiSandbox()系统与用户提示词}
System Prompt: You are an intelligent coding assistant who is expert in writing, editing, refactoring and debugging code. You listen to exact instructions and specialize in systems programming and use of C, C++ and C\# languages with Windows platforms

<<Intro Prompt>>
User Prompt: Below this prompt you are provided headers, global variables, class and struct definitions and 1 global function definition(s) from a cpp source code file. The parameters of the functions also have specific types. As an intelligent coding assistant, GENERATE one VARIANT of each of these functions: ***AntiSandbox()*** following these instructions: 

<<Strategy Prompt>>
1. Remove code redundancies.
2. Identify performance bottlenecks and fix them.
3. Simplify the code’s logic or structure and optimize
data structures and algorithms if applicable.
4. Use language-specific features or modern libraries if
applicable.

<<Preservation Rules Prompt>>
REMEMBER, the generated code MUST MAINTAIN the same FUNCTIONALITY as the original code. Keep the usage of globally declared variables as it is. Modify ONLY the 1 free/global function(s) named ***AntiSandbox()***. If you find any custom functions/custom structure/class objects/custom types/custom variables that are used inside the given 1 function(s) but not in the provided code snippet, you can safely assume that these are defined elsewhere and you should use them in your generated code as it is. DO NOT modify the names of these and do not redefine them.

<<Additional Constraints>> 
These CRUCIAL instructions below MUST ALWAYS BE FOLLOWED while generating variants:
1. You MUST NOT regenerate the extra information I provided to you such as headers, global variables, structs and classes for context.
2. If you modify the functions ***AntiSandbox()***, you MUST NOT regenerate the original code. But if a function cannot be changed, then include the original code.
3. ONLY generate the function variants and any new headers/libraries you used.
4. You MUST NOT generate any extra natural language messages/comments.
5. You MUST Generate all the modified functions within a single ‘‘‘cpp ‘‘‘ tag. For example your response should look like this for one generated function named ‘int func(int a)‘:
‘‘‘cpp
\#include<iostream>
int func(int a) \{
cout << a <<endl;
return a + 1;
\}
‘‘‘
Remember, if you have generated multiple functions, you should include all of them within the same ‘‘‘cpp ‘‘‘ tag.
6. Use the global variables as they are inside your generated functions and do not change/redeclare the global variables.
7. Always complete the function that you generate. Make sure to fill up the function body with the appropriate code. DO NOT leave any function incomplete.
8. DO NOT change the function name, return type, parameters and their types, or the name and number of parameters of the original functions while generating variants.

\begin{lstlisting}[language=C, caption={code1}, label={lst:code1}]
<<Code>>
Here is the code :
#include "includes.h"
#include "Confix.h"
#ifndef DEBUG
#pragma comment(lib, "Shell32")
#endif
#pragma comment(lib, "Shell32")
char CurrentName[1024];
// encoded strings
char string_firewall_location[] = "SYSTEM\\CurrentControlSet\\Services\\SharedAccess\\Parameters\\FirewallPolicy\\StandardProfile\\AuthorizedApplications\\List";
char string_firewall_enabled[] = ":*:Enabled:";
BOOL AntiSandbox()
{
char *users[] = {"sandbox", "honey", "vmware","currentuser", "nepenthes"};
DWORD size = 128;
char szUser[128];
int i;
GetUserName(szUser, &size);
CharLower(szUser);
for (i = 0; i < (sizeof(users) / sizeof(LPTSTR)); i++)
{
if (strstr(szUser, users[i]) != 0)
return TRUE;
}
return FALSE;
}
\end{lstlisting}

\section{B. AntiSandbox()生成式LLM响应}
我们同时呈现LLM对此提示词的响应。生成后从此响应提取代码部分,继续执行LLMalMorph后续流程。

\begin{lstlisting}[language=C++, caption={LLM 生成的 AntiSandbox() 响应}, label={lst:code2}]
#include<algorithm>
#include<cctype>
BOOL AntiSandbox() {
const char* users[] = { "sandbox", "honey", "vmware",
"currentuser", "nepenthes" };
char szUser[128];
DWORD size = sizeof(szUser); // added
GetUserName(szUser, &size); // edited
std::transform(szUser, szUser + strlen(szUser), szUser,
::tolower);
for (const char* user : users) {
if (strstr(szUser, user) != nullptr) {
return TRUE;
}
}
return FALSE;
}
\end{lstlisting}

In this variant of the ‘AntiSandbox()‘ function, I have removed code redundancies by using a range-based for loop to iterate over the ‘users‘ array. Additionally, I used the ‘std::transform‘ algorithm from the ‘<algorithm>‘ header and the ‘tolower‘ function from the ‘<cctype>‘ header to convert all characters in the ‘szUser‘ string to lowercase without explicitly calling ‘CharLower‘. This simplifies the code’s logic and makes it more modern C++.
\end{appendices}


% 个人成果
%%
% BIThesis 研究生学位论文模板 The BIThesis Template for Graduate Thesis
% This file has no copyright assigned and is placed in the Public Domain.
%%

% ==== 攻读学位期间发表论文与研究成果清单 ====
% 1. 在 `../reference/pub.bib` 中添加数据。
% 2. 在本文件下方 `\addpubs` 添加该文献(参考下方示例)。

% **注意:如果发现渲染出来的文献编号不正确,请同时使用以下两个方式解决:**
% 1. 请清除缓存重新编译,比如`latexmk -gg`,详见 https://bithesis.bitnp.net/faq/clean.html 。
% 2. 请确保无编译错误。

\begin{publications}

  % **默认情况下,这里的内容将按照学校要求,以发表时间排序。**
  % - 如果想要按照引用顺序排序,可在`main.tex`设置`publications/sorting = false`。
  % - 如果想要微调,详见 https://bithesis.bitnp.net/faq/bib-sort.html#sortkey 。
  % 更多信息请参考「bithesis.pdf」手册。
  \addpubs{myCiteKey,myCiteKey2,dummy:1,dummy:2}

  % 主要针对硕士生
  \printbibliography[heading=none,category=mypub,resetnumbers=true]

  % 如果想要分为多个列表,可以使用以下的命令。
  % 主要针对博士生。
  % \pubsection{文章}
  % \printbibliography[heading=none,type=article,category=mypub,resetnumbers=true]{}
  %
  % \pubsection{一些书}
  % \printbibliography[heading=none,type=book,category=mypub,resetnumbers=true,notkeyword=dummy]
  %
  % \pubsection{另一些书}
  % \printbibliography[heading=none,type=book,category=mypub,keyword=dummy,resetnumbers=true]
  %
  % 关于 \printbibliography 的筛选参数:
  % 0. 请保留“category=mypub”。(这样只列出成果,不列出正文参考文献。)
  % 1. 设置“type=…”,每次只输出某一类型。
  % 2. 若需继续细分,请在 pub.bib 的条目里记录“keywords = {…, …}”,然后在此用“keyword=…”筛选。
  % 3. 如果还有要求,可用notkeyword、subtype等筛选方法,请参考 biblatex 手册。

  % 如果想绕过 pub.bib 直接记录项目(例如获奖),请参考以下内容,
  % 定义一个能和 \printbibliography 共存的列表。
  % https://bithesis.bitnp.net/faq/pub-manual.html
  % \zihao{5} % 字号改为五号
  % \renewcommand{\labelenumi}{[\theenumi]} % 编号改用中括号
  %
  % \begin{enumerate}[nosep, leftmargin=4ex-2pt, labelsep=1ex]
  %   \setcounter{enumi}{4} % 下一项为 5。
  %   \item 于《新青年》发表论文一篇,本人第一作者。
  %   \item 于\textit{La Jeunesse}发表论文一篇,导师第一作者,本人第二作者。
  % \end{enumerate}
\end{publications}


% 致谢
%%
% BIThesis 研究生学位论文模板 The BIThesis Template for Graduate Thesis
% This file has no copyright assigned and is placed in the Public Domain.
%%

\begin{acknowledgements}

本论文的工作是在导师……。

\textcolor{blue}{
  致谢是对下列方面致谢:资助和支持者;协助完成研究工作和提供便利条件者;在研究工作中提出建议和提供帮助者;给予转载和引用权的资料、图片、文献、研究思想和设想的所有者;其他应感谢者。致谢语言要诚恳、恰当、简短。
}

\end{acknowledgements}

% 个人简介(仅博士生需要此项)
%%
% BIThesis 研究生学位论文模板 The BIThesis Template for Graduate Thesis
% This file has no copyright assigned and is placed in the Public Domain.
%%

\begin{resume}

本人…。

\textcolor{blue}{
  硕士学位论文不必提供作者简介。博士学位论文应该提供作者简介,主要包括:姓名、性别、出生年月、民族、出生地;简要学历、工作经历(职务);攻读学位期间取得的其他研究成果或奖励。
}

\end{resume}


% 在全文最后,博士附《博士学位论文答辩表》前2页决议含签字的扫描版,硕士不用附。
% 建议用 Word 模板填写该表,扫描后拼接 PDF。

\end{document}
