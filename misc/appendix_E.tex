\chapter{机器学习模型与阈值详情}

本节详述机器学习模型的具体架构。Malconv主要采用卷积神经网络设计,通过将恶意软件作为原始字节流处理进行分类。ResNet50分类器底层使用原始ResNet50模型\parencite{He2016},该模型首先将恶意软件转换为灰度图像,再利用图像进行分类。而Malgraph模型不直接使用图像/可执行文件,它是一种分层图恶意软件分类器,采用两个基于图神经网络(GNN)的编码层:函数内层将单个函数的控制流图(CFG)编码为向量,函数间层则利用前层生成的向量及外部函数,对函数调用图(FCG)表征进行编码以学习全局程序表示。最终预测层对此嵌入向量应用多层感知机(MLP)计算恶意概率。

对于Malconv和MalGraph,我们采用文献\parencite{Ling2024}提供的现成实现,该模型基于文献\parencite{Ling2022}所引入的数据集训练——该数据集包含210,251个Windows可执行文件(101,641个恶意软件和108,610个正常软件),涵盖848个恶意软件家族。数据集构成与模型性能的补充细节详见\parencite{Ling2024}。针对ResNet50,我们采用预训练的ImageNet模型,基于文献\parencite{Li2025}最新提出的恶意软件图像表征进行微调,该数据集包含恶意软件样本(2024年3月、4月、5月、7月和8月从MalwareBazaar\footnote{https://bazaar.abuse.ch/}收集)及对应正常软件。训练数据包含7,312个恶意软件实例和14,338个正常软件实例,恶意软件与正常软件比例为0.5:1。此训练分类器的性能在独立测试集上评估,该测试集包含9月恶意软件(同样采集自Malwarebazaar)和正常软件样本,恶意软件与正常软件比例为0.44:1(1,337个恶意软件和3,020个正常软件)。分类器在此测试集达到85\%准确率和85\% F1分数,更多细节参见[43]。

对于所有三个分类器,我们采用文献\parencite{Ling2024}设定的0.1\%误报率(FPR)阈值。恶意软件检测中需要低误报率以减少对良性文件的误判,但这也提高了决策阈值,可能导致更高的攻击成功率。此权衡关系已在先前研究\parencite{Ling2024}中被指出。针对Malconv和MalGraph,文献\parencite{Ling2024}提供的现成实现已包含0.1\% FPR阈值。对于微调后的ResNet50模型,我们通过选择良性测试集(仅用于阈值校准而非评估)上恶意软件概率的99.9百分位点计算等效阈值。在此模型特定阈值下,Malconv和ResNet50均未标记10个原始恶意软件样本,仅Malgraph标记了Fungus、Dexter、Conti和Babuk。故而我们将对抗评估集中于这四个样本变体。