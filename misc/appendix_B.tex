\chapter{详细的恶意软件描述}
我们提供了为进行实验而选择的每个恶意软件样本的详细描述。我们还通过Triage Sandbox运行了所有样本,以从沙箱报告中了解它们的行为。对于两个样本,沙箱未提供有用信息,但我们添加了其余样本的描述,我们从沙箱中获得这些样本的信息。

Exeinfector

Exeinfector在相关的GitHub仓库\parencite{Cryptware2024}中被归类为感染型,并且VirusTotal为其标记了诸如持久化、长时间休眠、反调试(以及用户输入检测等行为。Triage沙箱报告显示了恶意活动,包括添加持久化运行项、修改注册表、在system32目录中释放文件以及执行系统语言发现。

Fungus

Fungus在相关的GitHub仓库\parencite{Cryptware2024}中被归类为通用犯罪软件,是一种复杂的多文件C++恶意软件。VirusTotal将其与ircbot和autorun7等家族标签相关联。它具有反沙箱技术、基于USB的传播、服务器通信、防火墙规避以及键盘记录能力。Triage报告指出其活动包括设置自动启动、加载DLL、执行释放的文件、修改注册表、执行系统语言和位置发现以及可疑地使用Windows API调用。

Dexter
Dexter是一种销售点木马,被识别为针对运行Microsoft Windows的POS机的恶意软件\parencite{Wikipedia2024Dexter}。它于2012年被发现,以从POS机窃取信用卡和借记卡信息、将收集的数据发送到预定服务器以及表现出类僵尸程序行为而闻名\parencite{MalwareBytes2024POS}。VirusTotal将Dexter分类为木马、勒索软件和下载器,带有poxters和dexter等家族标签。Triage报告指出它会删除自身、崩溃、通过修改注册表添加运行项以实现持久化、执行系统语言发现以及使用可疑的Windows API函数,如“AdjustPrivilegeToken”、“WriteProcessMemory”和“EnumerateProcess”

HiddenVNC Bot

HiddenVNC,根据其源代码readme文件开发于2021年,是一种多文件的复杂C++恶意软件。VirusTotal将其分类为木马和银行木马,将其与tinukebot、tinynuke和tinuke等家族标签相关联,表明它可能是以开后门、窃取信息和下载恶意文件而闻名的Tinynuke木马僵尸程序的一个实现\parencite{MalwareBytes2024Tinukebot}。该readme将其描述为一个隐藏虚拟网络计算工具,它创建一个“隐藏桌面”,允许攻击者在用户不知情的情况下控制目标机器。其功能包括控制多台机器、启动应用程序(例如浏览器、PowerShell)以及在隐藏桌面上执行远程命令。它生成两个可执行文件;我们使用VirusTotal检测率为76.503\%的Client.exe,因为它比Server.exe(15.277\%)更具恶意性。

Predator

Predator,也称为Predator the Thief,于2018年中首次被发现\parencite{Fortinet2019Predator}。该样本是一种复杂的C++信息窃取木马,从其受感染主机窃取广泛的数据,例如系统信息、存储的浏览器密码、cookie、表单数据,甚至加密货币钱包地址\parencite{Fortinet2019Predator}。它还能捕获网络摄像头照片、记录击键、从应用程序(例如VPN、FTP、游戏客户端)提取凭证,并收集剪贴板内容和加密货币钱包文件\parencite{Fortinet2019Trojan}。VirusTotal将其归类为木马,并与窃取器、adwarex和fragtor等家族标签相关联。Triage沙箱报告向我们显示,它读取FTP客户端存储的数据文件、读取Web浏览器的用户/配置文件数据、从不安全的文件中窃取凭证、访问加密货币钱包并可能执行凭据收集。

Prosto

Prosto,也称为ProstoStealer,同样是一种信息窃取木马,但比Predator样本大得多且复杂得多,并且是用C++编写的。它利用受害机器窃取有价值且有用的信息,如登录信息、凭证、密码和直接文件。所有细节都存储在由攻击者控制的服务器中,以便后续用于诈骗和欺诈活动\parencite{Spyware2020}。VirusTotal将此样本分类为木马、病毒和间谍软件,并用fragtor和convagent等家族标签进行标记。该样本的Triage沙箱报告告诉我们,它会检查计算机位置设置、读取Web浏览器的用户数据、修改Internet Explorer设置,并且还有可疑地使用“FindShellTrayWindow”、“SetWindowsHookEx”、“WriteProcessMemory”等方法。

Conti

Conti勒索软件于2019年末出现\parencite{Wikipedia2019Conti}。这是一种极其复杂的恶意软件,包含不同的活动部件,拥有超过8000行用C++编写的代码。在我们的实验中,我们使用该样本的cryptor可执行文件。它采用双重勒索策略,在窃取数据的同时加密文件,以迫使受害者支付赎金。它以其快速的加密速度和针对医疗保健等关键行业而闻名\parencite{Wikipedia2019Conti}。VirusTotal将该可执行文件分类为木马和勒索软件,并将其与conti和adwarex等家族标签相关联。

Babuk

Babuk,也称为Babyk,是一种复杂的勒索软件,于2021年初被发现\parencite{McAfee2019},针对多个平台,如Windows、适用于Linux的ARM以及VMware ESXI环境,并使用椭圆曲线算法构建加密密钥\parencite{Fraunhofer2021}。该勒索软件用C++编写,约4000行代码,极其复杂,针对多个国家和地区的医疗保健、塑料、运输、电子和农业等行业\parencite{McAfee2019}。我们在实验中使用了该勒索软件的加密模块,并使用了生成的“.bin”格式可执行文件。VirusTotal将其归类为勒索软件和木马,并用babuk、babyk和epack等家族标签进行标记。Triage报告向我们显示,它是一种babuk加密器并属于babuk家族,会删除自身的卷影副本、重命名多个文件并添加文件扩展名、枚举已连接的驱动器和物理存储设备、与卷影副本交互,并且还有各种可疑的Windows API使用。

RedPetya Ransomware

RedPetya属于Petya家族加密恶意软件,于2016年首次被发现\parencite{Wikipedia2016Petya}。它采用引导锁式加密,在感染受害者后覆盖系统的主引导记录并强制重启;此时不是加载Windows,而是显示一个虚假屏幕,同时恶意软件使用加密算法在磁盘上秘密加密NTFS主文件表\parencite{MalwareBytes2016}。我们使用了其源代码的一个开源版本,该版本用C++编写,约1500行代码,使用OpenSSL进行加密,是对RedPetya恶意软件的完全重写。VirusTotal将此样本分类为木马和勒索软件,并给出petya、heur3和diskcoder等家族标签。Triage报告还显示它具有持久性,是一个引导区病毒,并且会写入主引导记录。它还显示可疑地使用Windows API,如“EnumeratesProcesses”、“AdjustPrivilegeToken”。

RansomWar

该样本是一种相对简单的勒索软件,用C语言编写,共1377行代码,使用河豚加密算法加密文件,并且代码中内置了邮件发送机制。VirusTotal将其归类为木马、勒索软件,并给出barys、ransomware等家族标签。从Triage沙箱中,我们得知它会枚举已连接的驱动器并在System32目录中释放文件。


