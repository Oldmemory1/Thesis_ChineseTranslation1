\chapter{额外恶意软件检测率分析}
VirusTotal的Dexter分析

如图\ref{fig:5.1}中Dexter的第三子图所示,优化策略检测率持续下降,在六次函数修改后稳定在均值附近。当修改文件"injectSection"(负责进程代码注入和资源管理)中的第二个函数后,"复用性"策略检测率骤降至67.847\%。尽管该函数初始导致检测率下降,但随后回升至74.537\%。其余策略检测率则始终接近整体平均值。

VirusTotal的Prosto分析

如图\ref{fig:5.1}中Prosto窃取器样本的第六子图所示,检测率波动显著:复用性策略在第4至第6次修改间急剧下降,优化策略则在第10至第11次修改间出现陡降。优化、Windows及安全策略均呈下降趋势,其中优化策略最低达52.738\%,较基线62.033\%降低9.295\%。LLM采用替代性Windows API函数进行Base64编码和HTTP连接管理可能导致此下降。

VirusTotal的Babuk分析

如图\ref{fig:5.1}中Babuk勒索软件样本的第八子图所示,优化策略在第二个函数修改起始处出现显著下降,检测率达64.861\%,较基线71.759\%降低近7\%。该变体在此策略的分数微升但仍低于基线检测率。除"复用性"外所有策略均呈现类似趋势,但检测率下降幅度不大。

混合分析的Predator分析

如图\ref{fig:5.2}中Predator窃取器样本的第五子图所示,除Predator子图中少量偏移数据点外,多数变体呈现极小波动。值得注意的是,优化策略检测率略低于其他变体。

混合分析的Prosto分析

如图\ref{fig:5.2}中Prosto窃取器样本的第六子图所示,多数变体聚集在72.33\%的基线检测率附近且偏差较小。未发现任何特定策略变体呈现显著低于基线的检测率。

混合分析的Conti分析

在Conti勒索软件的分析图中,71.568\%的平均检测率较基线79.333\%低约8\%。优化策略在第5至第6个函数修改处出现急剧下降,后续函数检测率维持在65\%左右。此外,质量策略的检测率同样呈现下降趋势。