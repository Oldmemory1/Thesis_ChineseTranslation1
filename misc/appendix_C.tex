\chapter{用于LLM修改的文件及恶意软件函数选择数量}

本节阐述我们从每个恶意软件样本中选择待修改函数数量的方法。表~\ref{tab:4.1}虽概述了各恶意软件样本的函数总量,但我们的选择过程需经严格筛选以确保修改的有效性。我们排除了属于外部库(如加密库和头文件)的文件,仅聚焦于包含自定义恶意软件代码的文件,并仅考虑按算法~\ref{alg:Function Transformation Using LLM}顺序修改全局函数。

鉴于不同恶意软件样本的函数数量存在差异,我们采用系统化策略:对函数较少的样本修改更大比例,而对函数数量显著的样本则修改较小比例。该方法在保证充分修改覆盖率与LLMalMorph所需可控人工调试工作量之间取得了平衡。

应用该方法,我们选取并修改了如下函数:
\begin{table}[htb]
	\centering
	\caption{修改的函数选择标准}
	\label{tab:function_selection_criteria_for_modification}
    \begin{tabularx}{\textwidth}{@{} 
    >{\RaggedRight}p{0.5\textwidth} 
    >{\RaggedRight}p{0.5\textwidth} 
    @{}}
		\toprule
		函数数量 & 修改率 \\
		\midrule
		\textless 10  & 100\% \\
		10 - 20 & 60\% \\
        20 - 40  & 30\% \\
		40 - 70 & 20\% \\
        \textgreater 70  & 15\% \\
		\bottomrule
	\end{tabularx}
\end{table}

应用此方法,我们选择并修改了以下函数:
\begin{table}[htb]
	\centering
	\caption{每个恶意软件样本的函数选择}
	\label{tab:function_selection_for_each_malware_sample}
    \begin{tabularx}{\textwidth}{@{} 
    >{\RaggedRight}p{0.25\textwidth} 
    >{\centering}p{0.25\textwidth} 
    >{\centering}p{0.25\textwidth} 
    >{\RaggedRight}p{0.25\textwidth} 
    @{}}
		\toprule
		恶意软件样本 & 总共选择的函数 & 修改的函数 & 修改率 \\
		\midrule
		Exeinfector & 4 & 4 & 100\% \\
		Fungus & 46 & 9 & 20\% \\
        Dexter & 61 & 12 & 20\% \\
		HiddenVNC bot & 60 & 12 & 20\% \\
        Predator & 30 & 9 & 30\% \\
        Prostostealer & 70 & 14 & 20\% \\
		Conti ransomware & 93 & 14 & 15\% \\
        Babuk ransomware & 35 & 11 & 30\% \\
		RedPetya & 15 & 9 & 60\% \\
        ransomware & 9 & 9 & 100\% \\
		\bottomrule
	\end{tabularx}
\end{table}

在必要处我们进行向上取整。对于RansomWar样本,我们最初尝试修改包含4个函数的"blowfish.c"文件未能成功,这源于LLM即使在纠错机制下仍存在生成函数变体的限制。因此,我们将修改转向包含9个函数的"RansomWar.c"文件。由于该文件属于函数数量少于10的样本类别,我们修改了其全部函数。总体而言,这种结构化方法使我们能在保持一致性的同时,确保不会在函数数量庞大的样本(例如Conti勒索软件)中修改过多函数——这考虑到LLMalMorph所需的人工调试需求。